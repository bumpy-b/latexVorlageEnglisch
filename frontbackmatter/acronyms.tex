%!TEX root = ../dokumentation.tex

\chapter*{Abbreviations}
%nur verwendete Akronyme werden letztlich im Abkürzungsverzeichnis des Dokuments angezeigt
%Verwendung: 
%		\ac{Abk.}   --> fügt die Abkürzung ein, beim ersten Aufruf wird zusätzlich automatisch die ausgeschriebene Version davor eingefügt bzw. in einer Fußnote (hierfür muss in header.tex \usepackage[printonlyused,footnote]{acronym} stehen) dargestellt
%		\acs{Abk.}   -->  fügt die Abkürzung ein
%		\acf{Abk.}   --> fügt die Abkürzung UND die Erklärung ein
%		\acl{Abk.}   --> fügt nur die Erklärung ein
%		\acp{Abk.}  --> gibt Plural aus (angefügtes 's'); das zusätzliche 'p' funktioniert auch bei obigen Befehlen
%	siehe auch: http://golatex.de/wiki/%5Cacronym
%
\begin{acronym}[YTMMM]
\setlength{\itemsep}{-\parsep}


\acro{AGPL}{Affero GNU General Public License}
\acro{AODV}{Ad-hoc On-Demand Distance Vector}
\acro{API}{Application Programming Interface}
\acro{DSR}{Dynamic Source Routing}
\acro{HTML}{HyperText Markup Language}
\acro{IEEE}{Institute of Electrical and Electronics Engineers}
\acro{LLC}{Logical Link Control }
\acro{MAC}{Medium Access Control}
\acro{MANET}{Mobile wireless Ad-hoc NETwork}
\acro{PHY}{Physical Layer}
\acro{QoS}{Quality of Service}
\acro{SPIN}{Sensor Protocol for Information via Negotiation}
\acro{WPAN}{Wireless Personal Area Network}
\acro{WMN}{Wireless Mesh Network}
\acro{WSN}{Wireless Sensor Network}
\acro{WYSIWYG}{What You See Is What You Get}
\end{acronym}
