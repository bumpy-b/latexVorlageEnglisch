%
% Nahezu alle Einstellungen koennen hier getaetigt werden
%
%
\RequirePackage[l2tabu, orthodox]{nag}	% weist in Commandozeile bzw. log auf veraltete LaTeX Syntax hin

\documentclass[%
	oneside,        	% Einseitiger Druck.
	12pt,           		% Schriftgroesse
	parskip=half,		% Halbe Zeile Abstand zwischen Absätzen.
%	topmargin = 10pt,	% Abstand Seitenrand (Std:1in) zu Kopfzeile [laut log: unused]
	headheight = 12pt,	% Höhe der Kopfzeile
%	headsep = 30pt,	% Abstand zwischen Kopfzeile und Text Body  [laut log: unused]
	headsepline,		% Linie nach Kopfzeile.
	footsepline,		% Linie vor Fusszeile.
	abstracton,		% Abstract Überschriften
	english,        		% Translator
	DIV=calc,		% Satzspiegel berechnen
	BCOR=8mm,		% Bindekorrektur links: 8mm
	headinclude=true,	% Kopfzeile in den Satzspiegel einbeziehen
	footinclude=false,	% Fußzeile nicht in den Satzspiegel einbeziehen
	listof=totoc,		% Abbildungs-/ Tabellenverzeichnis im Inhaltsverzeichnis darstellen
	toc=bibliography,	% Literaturverzeichnis im Inhaltsverzeichnis darstellen
]{scrreprt}	% Koma-Script report Klasse , fuer laengere Bachelorarbeiten alternativ auch: scrbook


%%%%%%% Package Includes %%%%%%%
%%%%%%% 
\usepackage[margin=2.5cm,foot=1cm]{geometry}	% Seitenränder und Abstände
\usepackage[activate]{microtype}	% Zeilenumbruch und mehr, für besseren Zeichensatz
%\usepackage{titlesec}			% für Veränderung von Titel/Überschriften-Darstellung (sollte jedoch nicht mit KOMA-Script kombiniert werden
\usepackage[utf8]{inputenc}		% Zeichencodierung
\usepackage[T1]{fontenc}
\usepackage[onehalfspacing]{setspace}% Zeilenabstand
\usepackage{makeidx}			% Index-Erstellung
\usepackage[english]{babel}		% Lokalisierung (englische Sprache)
\usepackage[babel,autostyle=true]{csquotes}% Anführungszeichen 
%\usepackage[babel,german=quotes]{csquotes}
%\usepackage[style=swiss]{csquotes}
\usepackage[right]{eurosym}		% Euro symbol: €
\usepackage{pgffor} 			% für automatische Kapiteldateieinbindung
\usepackage[perpage, hang, multiple, stable]{footmisc} % Fussnoten

% Spezielle Tabellenform für Deckblatt
\usepackage{longtable}
\setlength{\tabcolsep}{10pt} %Abstand zwischen Spalten
\renewcommand{\arraystretch}{1.5} %Zeilenabstand

% Grafiken
\usepackage{url,graphicx}
\usepackage{float} % for [H]
\usepackage{wrapfig}
\graphicspath{{images/}}% Bildpfad

% Alter some LaTeX defaults for better treatment of figures: (source: http://www-rohan.sdsu.edu/~aty/bibliog/latex/floats.html)
    % See p.105 of "TeX Unbound" for suggested values.
    % See pp. 199-200 of Lamport's "LaTeX" book for details.
    %   General parameters, for ALL pages:
    \renewcommand{\topfraction}{0.9}	% max fraction of floats at top
    \renewcommand{\bottomfraction}{0.8}	% max fraction of floats at bottom
    %   Parameters for TEXT pages (not float pages):
    \setcounter{topnumber}{2}
    \setcounter{bottomnumber}{2}
    \setcounter{totalnumber}{4}     % 2 may work better
    \setcounter{dbltopnumber}{2}    % for 2-column pages
    \renewcommand{\dbltopfraction}{0.9}	% fit big float above 2-col. text
    \renewcommand{\textfraction}{0.07}	% allow minimal text w. figs
    %   Parameters for FLOAT pages (not text pages):
    \renewcommand{\floatpagefraction}{0.7}	% require fuller float pages
	% N.B.: floatpagefraction MUST be less than topfraction !!
    \renewcommand{\dblfloatpagefraction}{0.7}	% require fuller float pages

	% remember to use [htp] or [htpb] for placement


% Mathematische Textsaetze
\usepackage{amsmath}
\usepackage{amssymb}

% Pakete um Textteile drehen zu können, oder eine Seite Querformat anzeigen kann.
%\usepackage{rotating}
%\usepackage{lscape}

% Farben
\usepackage{xcolor}
\usepackage{color}
\definecolor{LinkColor}{rgb}{0,0.1,0.3}
\definecolor{ListingBackground}{rgb}{0.9,0.9,0.9}

%\usepackage{draftwatermark} \SetWatermarkLightness{0}


% Glossar
\usepackage[
	nonumberlist, %keine Seitenzahlen anzeigen
	acronym,      %ein Abkürzungsverzeichnis erstellen
	%section,     %im Inhaltsverzeichnis auf section-Ebene erscheinen
	toc,          %Einträge im Inhaltsverzeichnis
]{glossaries}

% Abbreviations / Acronyms / Abkürzungen
\usepackage[printonlyused]{acronym} % falls gewünscht kann die Option footnote eingefügt werden, dann wird die Erklärung nicht inline sondern in einer Fußnote dargestellt

% Literaturverweise 
\usepackage[
	backend=biber,		% empfohlen. Falls biber Probleme macht: bibtex
	bibwarn=true,
	bibencoding=utf8,	% wenn .bib in utf8, sonst ascii
	sortlocale=en_US,
	style= numeric-comp,		% siehe auch unten
]{biblatex}

\addbibresource{bibliographie.bib}
%\addbibresource{weitereDatei.bib}

%% Zitierstil (nur wenn kein biblatex verwendet wird)
%% siehe: http://ctan.mirrorcatalogs.com/macros/latex/contrib/biblatex/doc/biblatex.pdf (3.3.1 Citation Styles)
%% mögliche Werte z.B numeric-comp, alphabetic, authoryear
%\bibliographystyle{unsrt}


% Verschiedene Schriftarten
%\usepackage{goudysans}
\usepackage{lmodern}
%\usepackage{libertine}
%\usepackage{palatino} 
\usepackage{courier} % Schriftart für monospaced

%%\usepackage{fontspec}		% requires XeTex
%\setsansfont{Calibri}
%\setmonofont{Consolas}

% Schriftart in Captions etwas kleiner
\addtokomafont{caption}{\small}


% Hurenkinder und Schusterjungen verhindern
% http://projekte.dante.de/DanteFAQ/Silbentrennung
\clubpenalty = 10000 % schließt Schusterjungen aus (Seitenumbruch nach der ersten Zeile eines neuen Absatzes)
\widowpenalty = 10000 % schließt Hurenkinder aus (die letzte Zeile eines Absatzes steht auf einer neuen Seite)
\displaywidowpenalty=10000


% Quellcode
\usepackage{listings}
%\usepackage{listing}
\usepackage{jvlisting}

\lstloadlanguages{PHP,Python,Java,C,C++,bash} % Einige häufig verwendete Sprachen

\lstset{%
	language=C,            % Sprache des Quellcodes
	numbers=left,           % Zeilennummern links
	stepnumber=1,            % Jede Zeile nummerieren.
	numbersep=5pt,           % 5pt Abstand zum Quellcode
	numberstyle=\tiny,       % Zeichengrösse 'tiny' für die Nummern.
	frame=1,
	breaklines=true,         % Zeilen umbrechen wenn notwendig.
	breakautoindent=true,    % Nach dem Zeilenumbruch Zeile einrücken.
	postbreak=\space,        % Bei Leerzeichen umbrechen.
	tabsize=2,               % Tabulatorgrösse 2
	basicstyle=\ttfamily\footnotesize, % Nichtproportionale Schrift, klein für den Quellcode
	showspaces=false,        % Leerzeichen nicht anzeigen.
	showstringspaces=false,  % Leerzeichen auch in Strings ('') nicht anzeigen.
	extendedchars=true,      % Alle Zeichen vom Latin1 Zeichensatz anzeigen.
	captionpos=b,            % sets the caption-position to bottom
	backgroundcolor=\color{ListingBackground} % Hintergrundfarbe des Quellcodes setzen.
	keywordstyle=\color[rgb]{0.133,0.133,0.6},
	commentstyle=\color[rgb]{0.133,0.545,0.133},
	stringstyle=\color[rgb]{0.627,0.126,0.941},
}

\lstset{%
%	language=C++, 
	basicstyle= \tiny \ttfamily, % \small
	frame=tb, 			% top bottom frame lines; alternative: single or 1
	columns=flexible, 
	showspaces=false,  	% show spaces adding special underscores
	showstringspaces=false,	% underline spaces within strings
	showtabs=false,               	% show tabs within strings adding particular underscores
	frame=lines,	                	% add a frame around the code
	tabsize=2,	                	% default tabsize: 4 spaces
	breaklines=true,                % automatic line breaking
	breakatwhitespace=false,	% automatic breaks should only happen at whitespace
	captionpos=b,            	% sets the caption-position to bottom
	numbers=left, 
	numberstyle=\tiny,
	keywordstyle=\color[rgb]{0.133,0.133,0.6},
	commentstyle=\color[rgb]{0.133,0.545,0.133},
	stringstyle=\color[rgb]{0.627,0.126,0.941},
	backgroundcolor=\color{ListingBackground} % Hintergrundfarbe des Quellcodes setzen.
}

\lstset{
	language=[Visual]C++,
	keywordstyle=\bfseries\ttfamily\color[rgb]{0,0,1},
	identifierstyle=\ttfamily,
	commentstyle=\color[rgb]{0.133,0.545,0.133},
	stringstyle=\ttfamily\color[rgb]{0.627,0.126,0.941},
	showstringspaces=false,
	basicstyle=\small,
	numberstyle=\footnotesize,
	numbers=left,
	stepnumber=1,
	numbersep=10pt,
	tabsize=2,
	breaklines=true,
	prebreak = \raisebox{0ex}[0ex][0ex]{\ensuremath{\hookleftarrow}},
	breakatwhitespace=false,
	aboveskip={1.5\baselineskip},
  	columns=fixed,
%  	upquote=true,		% requires textcomp package
  	extendedchars=true,
%	frame=single,
% 	backgroundcolor=\color{lbcolor},
}

% PDF Einstellungen
\usepackage[%
	pdftitle={\pdftitel},
	pdfauthor={\autor},
	pdfsubject={\arbeit},
	pdfcreator={pdflatex, LaTeX with KOMA-Script},
	pdfstartview=FitH,
	pdfhighlight=/I,
	pdfpagemode=UseOutlines, % Beim Oeffnen Inhaltsverzeichnis anzeigen
	pdfdisplaydoctitle=true, % Dokumenttitel statt Dateiname anzeigen.
	pdflang=eng % Sprache des Dokuments.
]{hyperref}

% (Farb-)einstellungen für die Links im PDF
\hypersetup{%
	colorlinks=true, % Aktivieren von farbigen Links im Dokument
	linkcolor=LinkColor, % Farbe festlegen
	citecolor=LinkColor,
	filecolor=LinkColor,
	menucolor=LinkColor,
	urlcolor=LinkColor,
	raiselinks=true,		% raise up links (for HyperTeX backend; default: false).
%	linktocpage=true, 		% Nicht der Text sondern die Seitenzahlen in Verzeichnissen klickbar
	bookmarksnumbered=true % Überschriftsnummerierung im PDF Inhalt anzeigen.
}
% Workaround um Fehler in Hyperref, muss hier stehen bleiben
\usepackage{bookmark} %nur ein latex-Durchlauf für die Aktualisierung von Verzeichnissen nötig


% Titel, Autor und Datum
\title{\titel}
\author{\autor}
\date{\datum}
